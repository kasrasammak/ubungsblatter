%\documentclass[10pt,twoside,a4paper,twocolumn,landscape]{article}
\documentclass[12pt,a4paper]{article}
\usepackage[margin=3cm]{geometry}
\usepackage{fancyhdr}
\usepackage{amssymb}
\usepackage{amsthm}
\usepackage[utf8]{inputenc}
\usepackage{parskip}
\usepackage{mathtools}
\usepackage[english]{babel}
\usepackage{hyperref}
\usepackage{mathrsfs}
\newcommand{\myeq}[1]{\hfill{\refstepcounter{equation}(\theequation)\label{#1}}}

% change subsections to alphabetic
\renewcommand\thesubsection{\thesection.\alph{subsection}}

\newtheorem{satz}{Satz}[section] % part
\newtheorem{lemma}{Lemma}[section] % part

\newcommand{\D}[2]{#1^\prime(#2)}
\newcommand{\norm}[1]{\left\lVert#1\right\rVert}
\newcommand{\abs}[1]{\left|#1\right|}
\newcommand{\scalar}[2]{\langle#1, #2\rangle}
\newcommand{\starr}{\star\star}
\newcommand{\starrr}{\star\star\star}
\newcommand{\starrrr}{\star\star\star\star}
\newcommand{\starrrrr}{\star\star\star\star\star}
\DeclareMathOperator{\Ima}{Im}
\DeclareMathOperator{\Bor}{Bor}
\DeclareMathOperator{\BorRd}{Bor(\mathbb{R}^d)}
\DeclarePairedDelimiter\ceil{\lceil}{\rceil}
\DeclarePairedDelimiter\floor{\lfloor}{\rfloor}
\renewcommand\qedsymbol{$\blacksquare$}

\begin{document}

\pagestyle{fancy}
\fancyhf{}
\rhead{Kasra Sammak}
\lhead{Analysis III - Übungsblatt 5}
\rfoot{\thepage}

\section*{Aufgabe IV.10.34}
Seien $\mathbb{R}_+ := [0, \infty],$ $f_n: \mathbb{R_+} \rightarrow \mathbb{R},$ $f_n := \frac{\sin(e^x)}{1+nx^2}$\\
\begin{align*}
	 &\sin(e^x), 1, x^2\text{ stetig }\xRightarrow[IV.4.3]{IV.4.2(c)}f_n \text{ messbar} &\myeq{}\\
	&\forall x \in \mathbb{R_+}: \lim_{n\to\infty} f_n = 0 =: f \text{ punktweise, da } \sin(e^x), 1, x^2 \text{ fest, und } \frac{1}{n} \xrightarrow{n \rightarrow \infty} 0 \ &\myeq{}\\
	\Rightarrow &\forall x \in \mathbb{R_+}: |f_n| \geq |f_{n+1}| \text{ punktweise}.\\
	&\forall x \in \mathbb{R_+}: |\sin(e^x)| \leq 1\Rightarrow |f_n| \leq \frac{1}{1+nx^2} \forall n \in \mathbb{N}\\
	\Rightarrow &\text{ Wähle } g := \frac{1}{1+x^2} \emph{ (die oberen Schrank von $f_1$) }\\& \text{ Dann ist $g$ integrierbar und } |f_n| \leq g \ \forall n \in \mathbb{N} \ &\myeq{} \\
	\xRightarrow[IV.6.2]{(1), (2), (3)} &\int_\mathbb{R_+}|f_n - f|dx = \int_\mathbb{R_+}|f_n - 0|dx = \int_\mathbb{R_+}|f_n|dx \xrightarrow{n \rightarrow \infty} 0 &\qed
\end{align*}
\section*{Aufgabe IV.10.35}
\begin{align*}
\hat{f}(y) = \frac{1}{\sqrt{2\pi}}\int_\mathbb{R} e^{-ixy}f(x)dx, \ f: \mathbb{R} \rightarrow \mathbb{C} \ \lambda\text{-integrierbar}	
\end{align*}
\subsection*{Warum ist $\hat{f}$ wohldefiniert?}
\begin{align*}
	\forall x,y \in \mathbb{R}: & e^{-ixy} \ \lambda\text{-integrierbar}\\ \forall x \in \mathbb{R}: &f(x) \ \lambda\text{-integrierbar}\\
	&\text{ Seien } \ y_1, y_2 \in \mathbb{R}  \text{ beliebig, mit } y_1 = y_2\\
	\xRightarrow{(*)}& \frac{1}{\sqrt{2\pi}}\int_\mathbb{R}e^{-ix{y_1}}f(x)dx = \frac{1}{\sqrt{2\pi}}\int_\mathbb{R}e^{-ix{y_2}}f(x)dx\\
	\Rightarrow &\hat{f}(y_1) = \hat{f}(y_2)
\end{align*}
(*) Das Integral ist nur auf $y$ abhängig, da es auf ganzen $\mathbb{R}$ über $x$ integriert. So ist $\hat{f}$ auch nur auf $y$ abhängig.
\subsection*{Ist $\hat{f}$ stetig?}
Da $e^{-ixy} \  2\pi$-periodisch ist:
\begin{align*}
	&\frac{1}{\sqrt{2\pi}}\int_\mathbb{R}e^{-ixy}f(x)dx = \frac{1}{\sqrt{2\pi}}\int_\mathbb{R}e^{-2\pi ixy}f(x)dx \ \forall x,y \in \mathbb{R} \ (y \ punktweise)\\
	\xRightarrow[\forall x,y \in \mathbb{R}]{|e^{-ixy}| \leq 1}&\left\lvert\frac{1}{\sqrt{2\pi}}\int_\mathbb{R}e^{-2\pi ix(y+h)}f(x)dx - \frac{1}{\sqrt{2\pi}}\int_\mathbb{R}e^{-2\pi ixy}f(x)dx\right\lvert  \\
	=& \left\lvert \frac{1}{\sqrt{2\pi}}\int_\mathbb{R}(e^{-2\pi ix(y+h)}- e^{-2\pi ixy})f(x)dx\right\lvert \\
	\leq& \frac{1}{\sqrt{2\pi}}\int_\mathbb{R}\left\lvert (e^{-2\pi ixh}-1)\right\lvert \left\lvert f(x) \right\lvert dx \xrightarrow{h \rightarrow 0} 0&\myeq{}\\
	&\text{Man kann } h := \frac{2\pi}{n}\text{ wählen, und die Folge messbarer (da integrierbar) } \\
	&\hat{f_n}: = \frac{1}{\sqrt{2\pi}} e^{-ixh}f(x) \text{ konstruieren, sodass }\\ 
	& \forall n \in \mathbb{N}: \lim_{n\to\infty} = \frac{1}{\sqrt{2\pi}}\int_\mathbb{R} e^{-ixh}f(x)dx = \hat{f} \text{ existiert.}&\myeq{}\\ 
	\xRightarrow[IV.6.2]{(4),(5)}&\left\lvert \frac{1}{\sqrt{2\pi}}\int_\mathbb{R}(e^{-2\pi ix(y+h)}- e^{-2\pi ixy})f(x)dx\right\lvert \xrightarrow{h \rightarrow 0} 0\\
	\Rightarrow &\hat{f} \text{ stetig, sogar gleichmäßig.} &\qed
\end{align*}
\section*{Aufgabe IV.10.36}
\subsection*{Sei $f: \mathbb{R} \rightarrow \mathbb{R}$ durch $f(s) = \frac{1}{\sqrt s}$ für $0 \textless s \leq 1$ und $f(s) = 0$ sonst definiert. Zeige, dass $f$ messbar ist, und berechne $\int_\mathbb{R} f d\lambda$.}
Seien $A: = (0,1]$ und $B:= (-\infty,0] \cap (1,\infty)$
\begin{align*}
	&f_A: (0,1] \rightarrow \mathbb{R} \text{ stetig } \xRightarrow{IV.4.2(c)} f_A\  \mathscr{B}orel-messbar \text{ auf } A\\
	&f_B: (-\infty,0] \cap (1,\infty) \rightarrow \mathbb{R} \text{ stetig } \xRightarrow{IV.4.2(c)}  f_B \ \mathscr{B}orel-messbar \text{ auf }  B\\
	\Rightarrow & f:\mathbb{R} \rightarrow \mathbb{R} \ \mathscr{B}orel-messbar 
\end{align*}
\begin{align*}
	\int_\mathbb{R} f(s) d\lambda(s)&=\int_A f(s) d\lambda(s) + \int_B f(s) d\lambda(s)\\
	&= \int_A \frac{1}{\sqrt s}\ d\lambda(s)+ \int_B 0\ d\lambda(s)\\
	&= 2\sqrt{s} \mid_{0}^{1} + \ 0\\
	&=2 - 0 \\
	&=2
\end{align*}
\end{document}
